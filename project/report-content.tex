\documentclass{article}

%install your packages here 
\usepackage[utf8]{inputenc}
\usepackage{amssymb}
\usepackage{amsmath}
\usepackage{mathtools}
\usepackage{amsthm}
\usepackage{setspace}
\usepackage{cancel}
\usepackage{bbm}
\usepackage{pgfplots}
\usepackage{listings}
\usepackage{multicol}
\usepackage{diagbox}
\usepackage{ tipa }
\usepackage{hyperref}
\usepackage{float}
\hypersetup{
    colorlinks=true,
    citecolor=black,
    filecolor=black,
    linkcolor=black,
    urlcolor=black
}

\usetikzlibrary{arrows, calc, patterns, shapes}
  \pgfplotsset{compat=1.15}
\renewcommand{\baselinestretch}{1.25}
\newcommand\sbullet[1][.5]{\mathbin{\vcenter{\hbox{\scalebox{#1}{$\bullet$}}}}}

%theorems
\theoremstyle{definition}
\newtheorem{theorem}{Theorem}

\theoremstyle{definition}
\newtheorem{definition}{Definition}

\newtheorem{example}{Example}[section]

%write short hand notes here 
%standard stats
\def\E{\mathbb{E}}
\def\l{\ell}
\def\xs{\{x_1, \hdots, x_n\}}
\def\Xs{\{X_1, \hdots, X_n\}}

%standard cal
\def\j{\mathcal{J}}
\def\sumn{\sum^n_{i=1}}
\def\inv{^{-1}}
\def\w{\omega}
\def\R{\mathbb{R}}
\def\fish{\mathcal{I}}
\def\v{\vec{v}}
\def\V{\Vec{V}}

%symbols
\newcommand{\dotrel}[1]{\mathrel{\dot{#1}}}



\title{Math 806222 Project}
\author{Jean }
\date{Fall 2020}

\begin{document}
\maketitle
\tableofcontents{}
\pagebreak
\section{Introduction}
\begin{enumerate}
    \item more generalized gp models
    \item Functional forms 
    \item discuss data 
    \item discuss shortcomings of paper
    \begin{enumerate}
        \item no testing of model on unseen data (use the fact that we have precipitation data to evaluate how well model predicts next year's rain)
    \end{enumerate}
\end{enumerate}

\section{TODO}
\begin{enumerate}
    \item check if EGP is worse than GP at small $n$ (for many ev problems n is not huge so this is a good point to make
    \item report on importance of EGPs for threshold selection
    \item kappa stabilizing to 1 really means as u grows we converge to original gp.
\end{enumerate}

\section{Questions }
\begin{enumerate}
    \item 
\end{enumerate}
\end{document}